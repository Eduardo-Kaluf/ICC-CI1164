\documentclass[11pt]{article}

\title{Otimização de Desempenho para Sistemas Lineares Esparsos com Pré-condicionantes}
\author{SERGIO SIVONEI DE SANT'ANA FILHO GRR20242337\\EDUARDO KALUF GRR20241770}
\date{\today}

\usepackage[utf8]{inputenc}
\usepackage[T1]{fontenc}
\usepackage[brazil]{babel}
\usepackage{url}
\usepackage{hyperref}


% compile -> latexmk -pdf relatorio.tex
% clean   -> latexmk -C

\begin{document}
    \maketitle

    \section{Introdução}\label{sec:introducao}

    Este é o Relatório

    $int filled_cells = (k * (n - 2b)) + 3b^2 + band_width;$

    !!!
    IMPORTANTE, ATUALMENTE A MÉTRICA DO GRADIENTE CONJUGADO ESTÁ SENDO FEITA POR COMPLETO, CASO O GRAFICO REQUERIDO SEJA DA MÉDIA, DIVIDIR INFORMAÇOES PELO TANTO DE EXECUÇÕES    !!!

    \section{Otimizações}\label{sec:otimizacoes}

    \subsection{Gradiente Conjugado}\label{subsec:gradiente}

    \subsubsection{CSR matrix}\label{subsubsec:csr}

    \subsection{Residuo}\label{subsec:residuo}

    \subsection{Simetrica positiva - extra}\label{subsec:simetrica}


    \section{Resultados}\label{sec:resultados}




    \section{Conclusão}\label{sec:conclusao}


    \section{Referencias}\label{sec:referencias}


    % Exemplo de uma citação
    O formato CSR é amplamente discutido na literatura\cite{wiki_sparse}.

    \nocite{*}
    \bibliography{references}
    \bibliographystyle{plain}

\end{document}
